% Disclaimer: This document was modified from the form in which I received it in order to maximize its utility to lab report writers

\documentclass[11pt,oneside,letterpaper]{article} % <-- Could change the font size up here
\usepackage[margin=1.0in]{geometry}               % Packages and things to be imported that may be used
\usepackage[utf8]{inputenc}             	      % Later on in tables, figures, etc.

\usepackage{color}
\usepackage[colorlinks, linktocpage, linktoc=section, urlcolor=blue, linkcolor=blue, citecolor=red]{hyperref}

\usepackage{graphicx}				

\usepackage{amsmath}
\usepackage{amssymb}
\usepackage{float}
\usepackage{array}
\usepackage{subfigure}


\usepackage{appendix}


\newcommand{\UR}{University of Rochester}	% Custom commands


\title{Lab Report Template}
\author{\underline{Your Name} \and Partner Name}
\date{\today}

\begin{document} % Starts the Document 
\maketitle


\abstract % Latex has its own abstract heading!
Abstract section. Briefly outline what the report covers and state results if applicable.
To typeset math inline with text, use dollar signs on either side of your equation: $g = 9.81m/s^2$



\section{Theory}
Define sections and subsections to organize your report.


\subsection{Part 1}
You can add figures with descriptive captions and refer back to them later:
\newline % Go to the next line


\begin{figure}[H] % The [H] tells LaTeX to put your figure right were you want it.
   \centering
   \includegraphics[width=0.7\columnwidth]{Figures/Solvay_conference_1927.jpg} % This .tex document is in the same root directory as the folder
   \caption{The 5\textsuperscript{th} Solvay conference, 1927}
   \label{fig:solvay} % label for later reference
\end{figure}


\par Numbered equations can be constructed as % \par does paragraph indentation
\begin{equation}
	\vec{L} = I \vec{\omega} \label{eq:angMomentum} % Label figures and equations to dynamically refer back to them later. You can also label and refer to document sections.
\end{equation}


\section{Experiment}
You can use \verb|\autoref| to refer back to \autoref{fig:solvay} or \autoref{eq:angMomentum}.



\section{Data Analysis}
Might want a table to present data:


\begin{table}[H]
\centering
\begin{tabular}{|c|c|c|}
	\hline
	Trial & $\theta_1$ & $\theta_2$  \\ \hline
	1 & 1 & 2 \\ \hline
	2 & 1 & 2 \\ \hline
	\end{tabular}
	\caption{caption about table 1}
	\label{tbl:data}
\end{table}

I highly recommend using \href{http://www.tablesgenerator.com/latex_tables}{the {\LaTeX} table generator} to format your {\LaTeX} tables.
\newline

Of course, error propagation will be in here:
\begin{equation}
    \sigma = \sqrt{\left(\frac{\partial}{\partial x}\right)^2\sigma _x^2+\left(\frac{\partial}{\partial y}\right)^2\sigma _y^2 + \left(\frac{\partial}{\partial z}\right)^2\sigma _z^2\cdots} 
\end{equation}

Maybe some weighted mean:
\begin{align*} % Align equations in an align environment with the & character. A * after then environment name removes the equation numbers.
    w_i &= \frac{1}{\sigma_i^2}\\
    x_{\text{mean}} &= \frac{\sum\limits_{i=1}^N{w_ix_i}}{\sum\limits_{i=1}^N{w_i}}\\
    \sigma_{\text{mean}} &= \frac{1}{\sqrt{\sum\limits_{i=1}^N{w_i}}}
\end{align*}



\section{Conclusion}
Summary



\section{Remarks}
remarks



\end{document}  % Ends the masterpiece